\section{Raspberry Pi 2}
\textit{Raspberry Pi 2} ili \textit{RPi2} je jeftino, malo računalo (veličine kreditne kartice) koje je nastalo u edukacijske svrhe; kako bi ljudi što lakše mogli naučiti programiranje i što lakše razumjeti hardver. Po specifikacijama je nešto sporiji od običnih stolnih (\textit{Desktop}) računala. Koristi \textit{Raspbian Jessie} operativni sustav koji je baziran na \textit{Debian} operativnom sustavu; jednom od distribucija \textit{Linux-a}.\\

\begin{figure}[h]
    \centering
    \includegraphics[width=0.5\textwidth]{images/raspberry.jpg}
    \caption{Raspberry Pi 2 računalo}
    \label{fig:mesh1}
\end{figure}
\\
Jedna od najzanimljivijih stvari koje \textit{Raspberry Pi 2} posjeduje su GPIO \footnote{engl. General Putpuse Input/Output (GPIO) - pinovi koji su ugrađeni od strane proizvođača, i omogućuju definiranje pinova kao ulaza, odnosno kao izlaza, prema korisnikovim željama.
} pinove; koji omogućuju komunikaciju s ostalim ulazno/izlaznim komponentama hardvera (senzori, motori,..), kao i s ostalim \textit{Rasppbery Pi} računalima.\\
\\
Princip njihovog rada je naponska logika. Ukoliko je GPIO pin postavljen kao Izlazni pin; tada posjeduje dva stanja, logičku nulu (nema napona) odnosno logičku jedinicu (ima napona). Na sljedećoj slici je prikazan rapsored GPIO pinova. U ovom radu od interesa su GPIO pinovi; a \textit{Raspberry Pi 2} posjeduje sedamnaest takvih pinova.\\

\begin{figure}[h]
    \centering
    \includegraphics[width=0.5\textwidth]{images/pinout.png}
    \caption{Raspored GPIO pinova}
    \label{fig:mesh1}
\end{figure}
Za kontrolu pinova, odnosno postavljanje pinova kao ulaza ili kao izlaza, koristio se Python programski jezik; kao i biblioteka \texit{RPi.GPIO} \footnote{Biblioteka otvorenog koda, preinstalirana sa Raspbianom i Python kompajlerom. Vidjeti poglavlje OVDE STAVIT BROJ POGLAVLJA}
\newpage
Pinovi koje \textit{Raspberry Pi} posjeduje su:
\begin{itemize}
    \item Naponski pinovi (2 x 5V, 2 X 3.3V, 5 x GND)
    \item GPIO x 17\\
\end{itemize}

Uz GPIO pinove \texit{RPi 2} posjeduje sve potrebne ulaze kao i stolno računalo, pa tako posjeduje i četiri serijska porta; koje ćemo iskoristiti za daljnu komunikaciju s komponentama. \footnote{Komunikacija se ostvaruje preko USB porta sa Arduino Uno 
kontrolerom. Vidjeti stranicu ARDUINO UNO i SPAJANJE ARDUINA ZA KOMUNIKACIJU}\\
Za pristup terminalu \textit{Rpi 2} operativnog sustaav korišten je \textit{Putty} \footnote{O PUTIJU} programski alat. Te preko UTP kabela je ostvareno fizičko povezivanje. 