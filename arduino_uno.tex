\section{Arduino Uno}
\textit{Arduino Uno} je programabilni mikrokontroler baziran na ATmega328P\footnote{Jednoprocesorski mikrokontroler proizveden od strane Atmela}. Posjeduje četrnaest digitalnih ulazno/izlaznih pinova, šest analognih pinova, naponske pinove, te USB konekciju \footnote{MASTER SLAVE}. Na sljedećoj slici je prikazana \texit{Arduino Uno} pločica.
\begin{figure}[h]
    \centering
    \includegraphics[width=0.5\textwidth]{images/arduino.png}
    \caption{Mikrokontroler \textit{Arduino Uno}}
    \label{fig:mesh1}
\end{figure}

Za programiranje kontrolera koristio se Arduino IDE, razvojno okruženje otvorenog koda; koji olakšava pisanje koda i prebacivanje koda na kontroler.\\

Izgled osnovnog programa za Arduiono je sljedeći:\\


\begin{lstlisting}[caption={Struktura Arduino programa}, label=program]
void setup() {
  // put your setup code here, to run once:

}

void loop() {
  // put your main code here, to run repeatedly:

}
\end{lstlisting}\\


Programi za \textit{Arduino Uno} su podijeljeni u dvije funkcije prototipa \textit{void}. U prvoj funkciji \textit{void setup()} se kreiraju i inicijaliziraju varijable. Ova funkcija se poziva samo jednom prilikom startanja \textit{Arduino Uno} kontrolera. Druga funkcija \textit{void loop()} je funkcija koja se repetitivno ponavlja tokom rada kontrolera. Detaljniji rad programa za \texit{Arduino Uno} će biti pojašnjeno u DRIVERI ARDUINO.

