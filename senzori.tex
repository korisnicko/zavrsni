\subsection{Senzori HC - SR04}
Ultrazvučni senzori rade na principu "ispaljivanja" zvuka te osluškivanja njegove jeke. Na slici je prikazana shema spajanja senzora.
\\[\intextsep]
\begin{minipage}{\linewidth}
\centering%
\includegraphics[width=0.5\linewidth,clip=]{images/senzor_UZV.png}%
\figcaption{Shema spajanja ultrazvučnih senzora (HC-SR04)}%
\label{fig:xfce}%
\end{minipage}
\\[\intextsep]
\newline
\\[\intextsep]
\begin{minipage}{\linewidth}
\centering%
\includegraphics[width=0.8\linewidth,clip=]{images/puls.png}%
\figcaption{Princip detekcije prepreke ultrazvučnim valovima}%
\label{fig:xfce}%
\end{minipage}
\\[\intextsep]
Ultrazvučni senzor koristi dva \textit{GPIO} pina jedan kao \textit{Input (Echo)}, a drugi kao \textit{Output (Trig)}.
Kada se 8 segmentni ultrazvučni val "ispali" iz ultrazvučnog modula. Mjeri se vrijeme koje je potrebno da se val vrati natrag u modul nakon odbijanja od prepreke. S obzirom da je brzina zvuka 317 m/s (zrak), te vrijeme koje je bilo potrebno da se zvuk vrati  dolazimo do formule za udaljenost predmeta od senzora.
\newline

Udaljenost je veća od stvarne jer zvuk prevali dvostruki put, od senzora do prepreke i od prepreke do senzora, koje su jednake. S obzirom da je od zanimacije samo udaljenost između senzora i predmeta, rezultat se podijelio sa 2.
Formula za izračun udaljenosti bi bila
${Udaljenost=(vrijeme * brzina zvuka)/2}$
\newline
